\section{Literature Review}
\label{sec:literature-review}

We can divide the most important skills possessed by developers which are in demand by the hiring companies into two categories: \emph{Soft Skills} and \emph{Hard Skills}. Most Both of these skill groups are equally valuable in the eyes of companies and the most important ones are going as follows \cite{StackOverflowStudies}:

\begin{itemize}
  \item \textbf{Hard Skills:} Languages, Libraries \& Frameworks, OS \& Infrastructure, Process \& Methods, Data Systems, and Development Tools
  \item \textbf{Soft Skills:} Communication, Collaboration, Problem-Solving
\end{itemize}

These findings were based on job offers which were listed on \emph{Stack Overflow} website. That should be emphasized because this website is used primarily by developers for problem-solving and discussion - the job listing functionality is there as a side feature and that could lead possibly to results that are more technical.

Still, one of the most important factors is collaboration \cite{WhatMakesGoodDev} and even though it is indeed possible to dig out and assess technical skills as we have mentioned earlier, it was shown that there is a lack of any strong association between them and social skills \cite{WhatMakesGoodDev}. Although it could be complicated to obtain information about non-technical skills, there is possible to dig out this information's via GitHub Profile in conjunction with CV and Linked In\footnote{Linked In is an employment-oriented online service that operates via websites and mobile apps. The platform is mainly used for professional networking, and allows job seekers to post their CV's and employers to post jobs.} profile with satisfactory results \cite{CandidateSelection}. This is a small disclaimer but if ever it would be necessary to verify that the attached GitHub profile belongs to the applicant for whatever reason it is possible \cite{GitHubProfilesToJobAdv} mainly via \code{README.md} natural language analysation.

Surely, companies are also interested in the expertise of the potential candidates in a given set of languages and sometimes even precisely - frameworks. This is essential because hiring a developer in a senior position without real seniority is a very costly mistake given the current salaries for any senior position. There is also a lot of potential here, as it was shown that such information can be fairly successfully extracted with over $70\%$ (on average) success rate, particularly for distinguishing React, Node-Mongodb, and SOCKET.IO experts \cite{SoftwareLibraries}.

One would ask whether such a tool could work in practice as maybe there are too many completely irrelevant repositories that are not project-based. Luckily, this question was already addressed in the past and it turns out that not every repository is used just for programming purposes. Many of the repositories are abandoned after a few weeks - there is a substantial spike in activity in the first few weeks, but as time passes by, the activity is decreasing drastically (logarithmically to be exact). A good number of people are using repositories just as a storage place and there is also a noticeable amount of empty repositories as an effect of some experiments or learning attempts.

In fact, a quite similar tool was already developed \cite{GitLabProfilingTool} but it used \emph{GitLab} instead of \emph{GitHub} to aggregate user data which gives u a green light to pursue with the project. They managed analysed six conditions: \emph{code quality, code quantity, skills, contributions, personalized commit time, and project participation} and then used them to evaluate developer expertise. It was concluded that none of these could be chosen as the most significant and their importance was about equal. That is why our project is quite relevant as we address the similar, already detected issue, but we want to pursue with a much more popular service to hit bigger ,,audience'' of candidates \cite{GitLabvsGithub}.

We also recognize that there are few pitfalls that could lead to false-positive results, for example, if we take a deeper look into the on-boarding programs. Briefly, on-boarding programs rely on mentored work with open source contributions. It was demonstrated that this practice is not as effective at transitioning new developers into long-term contributors as it was previously speculated, although developers who after all do succeed through these programs are valuable. \cite{DoOnBoardingProgramsWork} That's why we are aware that not every contributor to the open-source is going to be a skilled programmer, even though he could receive many tips after being mentored by a professional, and moreover - he would most likely contribute with high-quality code which was not purely developed by the person behind the GitHub profile.

Nevertheless, such a tool is really in demand. The current recruitment process causes a lot of stress and is not liked among developers. Programmers say that during the interview they can not show their whole potential and they are served with exercises that are not useful during normal work and do not check their actual knowledge. Developers criticize big companies for their ease of rejecting candidates just because there are just so many more available candidates and even if they reject someone who is qualified, it will not really hurt the company because they will find somebody else. \cite{HiringIsBroken}
