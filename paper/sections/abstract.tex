\abstract*{
\newline
\noindent\textit{Context:} Nowadays, it is much harder for new programmers to acquire industry experience. This trend started being noticeable at around 2018, but now - due to the coronavirus pandemic - it is even more prominent. Several dozens of junior programmers apply for one job offer and thus there can be a need for a tool that would facilitate the identification of candidates that the employer is interested in. It is considered a good practice for Junior Developers that are seeking to be hired to have public repositories with some of their personal work. The problem is that usually the recruiter does not have the knowledge to use that information, which - in fact - could be very impactful as it is literally the "show-off" of one's current skills. Right now it is completely disregarded until the later stages of the recruitment phase - after the junior already passes the first recruiter look-up. Only then the repository is examined by a technical interviewer but that also could be a concern for the company as it books out the valuable time of an expertised employee. Moreover, there could be situations when one's CV is completely skipped due to various reasons and thus missing the chance to be employed or have his repositories examined by a professional which could prove otherwise (that one has all the  abilities and skills needed for the position).
\newline
\noindent\textit{Objective:}
This study aims to identify and investigate if its possible to acquire use-full information's from a GitHub profile that could be beneficial for recruiters in a recruitment process. If that would be proven to be an effective way of measuring one's abilities and skills - maybe it could be also used by the developers themselves to attest their own potential quality.
\newline
The visible main challenge is the inefficient hiring process with no accurate technical screening methodology. The goal is to improve the rate of recruiting highly skilled programmers by making use of the information's that are already provided in CV's by standard.
\newline
\noindent\textit{Method:} We conducted a systematic review using a database search in Scopus and evaluated it using the quasi-gold standard procedure to identify relevant studies.
We created polls to gather information from developers own experiences and asked them to attach their GitHub profiles to the questionnaire. Then, with machine-learning based approach, we used that data to automatically evaluate the potential of a candidate by giving only the link to GitHub profile on input.
In the data sheet used to obtain data from publications, we factor the research questions into finer-grained ones, which are then answered on a per-publication basis. Those are then merged over a set of publications using an automated script to obtain answers to the posed research questions.
\newline
\noindent\textit{Results:}
...
\newline
\noindent\textit{Conclusion:}
We conclude that based on the results, we achieved that there is enough data stored on \emph{GitHub} profiles from which machine learning algorithms could provide a reasonably reliable evaluation of the potential quality of the user as an employee, although more research with bigger sample sizes are needed to improve the credibility of these results.
}

\abstract{
\label{sec:abstract}
\newline
\noindent\textit{Context:} Nowadays, it is much harder for new programmers to acquire industry experience. This trend started being noticeable at around 2018, but now - due to the coronavirus pandemic - it is even more prominent. Several dozens of junior programmers apply for one job offer and thus there can be a need for a tool that would facilitate the identification of candidates that the employer is interested in. It is considered a good practice for Junior Developers that are seeking to be hired to have public repositories with some of their personal work. The problem is that usually the recruiter does not have the knowledge to use that information, which - in fact - could be very impactful as it is literally the "show-off" of one's current skills. Right now it is completely disregarded until the later stages of the recruitment phase - after the junior already passes the first recruiter look-up. Only then the repository is examined by a technical interviewer but that also could be a concern for the company as it books out the valuable time of an expertised employee. Moreover, there could be situations when one's CV is completely skipped due to various reasons and thus missing the chance to be employed or have his repositories examined by a professional which could prove otherwise (that one has all the  abilities and skills needed for the position).
\newline
\noindent\textit{Objective:}
This study aims to identify and investigate if its possible to acquire use-full information's from a GitHub profile that could be beneficial for recruiters in a recruitment process. If that would be proven to be an effective way of measuring one's abilities and skills - maybe it could be also used by the developers themselves to attest their own potential quality.
\newline
The visible main challenge is the inefficient hiring process with no accurate technical screening methodology. The goal is to improve the rate of recruiting highly skilled programmers by making use of the information's that are already provided in CV's by standard.
\newline
\noindent\textit{Method:} We conducted a systematic review using a database search in Scopus and evaluated it using the quasi-gold standard procedure to identify relevant studies.
We created polls to gather information from developers own experiences and asked them to attach their GitHub profiles to the questionnaire. Then, with machine-learning based approach, we used that data to automatically evaluate the potential of a candidate by giving only the link to GitHub profile on input.
In the data sheet used to obtain data from publications, we factor the research questions into finer-grained ones, which are then answered on a per-publication basis. Those are then merged over a set of publications using an automated script to obtain answers to the posed research questions.
\newline
\noindent\textit{Results:}
...
\newline
\noindent\textit{Conclusion:}
We conclude that based on the results, we achieved that there is enough data stored on \emph{GitHub} profiles from which machine learning algorithms could provide a reasonably reliable evaluation of the potential quality of the users behind them as employees, although more research with bigger sample sizes are needed to improve the credibility of these results.
}
