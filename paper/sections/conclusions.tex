\section{Conclusions}
\label{sec:conclusions}


\subsection{Further Research}

Collecting relevant data is difficult due to developers high understanding and concerns regarding privacy and data collection. Regardless, to improve the credibility of our analysis, one would have to gather larger sample data - we provide working tools and solutions which can help scrape the information and quickly start investigations, just data gathering is a difficult task in of it self.

Easier and more instantaneous way to improve on this paper would be by reproducing what we have done - the logic, tools and parsed data is already here and we believe much better results can be achieved by tweaking the parameters in the algorithms or for example, by using different classifications.


\subsection{Summary}

Our research is a solid brick in the foundation of automatic skill assessment based on the data that is already shared by  and available to use on version control websites. By no means - the task is not easy (which was also already concluded by similar research that focused on \emph{GitLab} rather than \emph{GitHub}) and we confirm on independently gathered data with our own methods and algorithms that the selection of criteria is still quite enigmatic and is not clear out.  

One of the main concerns could be that not every single user has some meaningful repositories available publicly on their \emph{GitHub} profile, but our analysis has disproved this as the majority of developers do have some kind of viewable personal projects on their \emph{GH}. Even more - majority of developers in the context of our research situation - that is - job search, should have some kind of work available to look up.

In that scenario, we conclude that the GitHub profile is able to provide a sufficient amount of data to assess the potential quality of a developer with reasonably reliable evaluation results.
