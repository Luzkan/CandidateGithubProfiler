\section{Introduction}

While hiring practices of graduates are not thoroughly studied in the context of
the IT industry, the concept in general is well understood
Selecting the most valuable candidate for interview is an important process. While hiring practices of graduates are not thoroughly studied in the context of the IT industry, the concept in general is well understood \cite{HiringProcess}. As a standard, recruiters look through Curriculum Vitae (\emph{CV}) to categorize all candidates and call them for interview. Nowadays, to minimize the time spent on a single CV, recruiters have various pre-screening methods, listing them in order of importance \cite{SantanderCVLectureScholarship}:
\begin{itemize}
  \item employment history and previously business-related work
  \item university and academic performance
  \item listed programming abilities and other colloquially called \emph{"buzzwords"}
  \item soft skills
\end{itemize}

The CV could include a lot of important information's but the recruiter usually spends from 6 to 60 seconds at max per one application \cite{SantanderCVLectureScholarship}. It is surely not enough time to examine candidate skills correctly, more over - the recruiters usually doesn't even have the industrial knowledge to asses the skills and examine candidate's quality by browsing through "code portfolio". It is done only after the pre-screening is already done, when the filtered-out list of candidates goes to a professional technical employee that has the required knowledge and skills to examine the potential of a candidate. The information is there - various online hosting websites that allow to showcase one's code are widely used and added to the CV's - but a qualified candidate may be omitted in the pre-screening method because his cv-building skills were not good enough to pass.

One of such hosting websites is GitHub \footnote{GitHub Website URL: https://github.com/} - it is a service supporting version control using git, which is used by programmers and developers for storing and managing theirs code-base repositories. The popularity of GitHub in 2021 is well established and still rising. Already in 2020 the website gathered over 36 million users \cite{GitHubUsers2020} and today, as of March 2021 - there are 55 million users \cite{GitHubUsers2021}. Dabbish et al. \cite{DabbishC} showed that software developers in public project on GitHub consciously manage their online reputation. They do recognise that others developers can assess their personal characteristics such as the quality of work or commitment by looking for example through their commits and judging the quality, checking the amount of forks of an project or stars. Singer et al. \cite{Singer} found that the amount of followers one have on GitHub is also an valuable information.

We recognise that the CV evaluation procedure has several shortcomings and not much has changed since 2013 \cite{Capiluppi}. These shortcomings especially apply for those who do not have any relevant work experience or formal degree. We also note that GitHub is a place storing lots of very valuable informations that could (or maybe should) be used to evaluate potential of a candidate. Various research works have confirmed that it is possible to retrieve data like technical technical role \cite{TechnicalRole}, identifying Experts in Software Libraries and Frameworks \cite{SoftwareLibraries} .Because the recruiters have very limited time available to review a single CV, we search for an machine-learning based solution that could help extract the information from GitHub by a non-expertised recruiter. This could not only reduce the amount of false-positive skips but also increase the overall quality of list of candidates after the pre-screening is done.
