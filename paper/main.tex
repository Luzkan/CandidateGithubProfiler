\documentclass[graybox]{svmult}

% |-------------------------------------------------------------------------------------
% | Packages.
% |

% |--------------------|
% | Template Packages
% |

\usepackage{bibentry}
\usepackage{type1cm}           % Activate if the above 3 fonts are not available on your system.

\usepackage{makeidx}           % Allows index generation.
\usepackage{graphicx}          % Standard LaTeX graphics tool when including figure files.

\usepackage{multicol}          % Used for the two-column index.
\usepackage[bottom]{footmisc}  % Places footnotes at page bottom.

\usepackage{newtxtext}
\usepackage{newtxmath}         % Selects TimesNewRoman as basic font.
\usepackage{dirtytalk} \newcommand{\mysay}[1]{\say{\textit{#1}}}
\usepackage{enumerate}
\usepackage[unicode,colorlinks=true,breaklinks,allcolors=black]{hyperref}
\usepackage{cleveref}
\usepackage{ltablex}
\usepackage{booktabs}
\usepackage{hyphenat}
\usepackage{makecell}
\usepackage{doi}
\usepackage{enumitem}

\usepackage{pgfplots}
\pgfplotsset{compat=1.17}
\usepgfplotslibrary{statistics}
\usetikzlibrary{pgfplots.statistics}

\usepackage{fixme}
\usepackage{fancyhdr}

% |--------------------|
% | Personal Packages
% |

% Pozwala na używanie kodu w papierze. // Marcel Jerzyk
\usepackage{listings}

% |-------------------------------------------------------------------------------------
% | Settings.
% |

\nobibliography*
\makeindex  % Used for the subject index, please use the style svind.ist with your makeindex program.
\graphicspath{{img/}}  % Default graphics path. // Marcel Jerzyk

% |-------------------------------------------------------------------------------------

\pagestyle{fancy}
\fancypagestyle{firstpage}{
  % Clears the default for header and footer
  \fancyhf{}
  \lhead{\footnotesize Preprint of a chapter: Marcel Jerzyk, Jakub Litkowski, Jakub Szańca and Lech Madeyski, Developments in Information and Knowledge Management for Business Applications, Studies in Systems, Decision and Control, chapter ”Title” Springer}
}
\setlength{\headheight}{110pt}

% |--------------------------------------------------------------------------------------
% | Beginning of the Document.
% | Use the package "url.sty" to avoid problems with special characters used in your e-mail or web address
% |

\begin{document}

\title*{Evaluating the potential of a candidate for a job offer based on his GitHub profile}
\titlerunning{Evaluating the potential of a candidate for a job offer based on his GitHub profile.}
\author{Marcel Jerzyk, Jakub Litkowski, Jakub Szańca and Lech Madeyski}
\authorrunning{M. Jerzyk, J. Litkowski, J. Szańca and L. Madeyski}
\institute{Marcel Jerzyk, Jakub Litkowski, Jakub Szańca, Lech Madeyski \at Wroclaw University of Science and Technology, Poland}

\maketitle

\thispagestyle{firstpage}

% |-------------------------------------------------------------------------------------
% | Abstract.
% |

\abstract*{
\newline
\noindent\textit{Context:} Nowadays, it is much harder for new programmers to acquire industry experience. This trend started being noticeable at around 2018, but now - due to the coronavirus pandemic - it is even more prominent. Several dozens of junior programmers apply for one job offer and thus there can be a need for a tool that would facilitate the identification of candidates that the employer is interested in. It is considered a good practice for Junior Developers that are seeking to be hired to have public repositories with some of their personal work. The problem is that usually the recruiter does not have the knowledge to use that information, which - in fact - could be very impactful as it is literally the "show-off" of one's current skills. Right now it is completely disregarded until the later stages of the recruitment phase - after the junior already passes the first recruiter look-up. Only then the repository is examined by a technical interviewer but that also could be a concern for the company as it books out the valuable time of an expertised employee. Moreover, there could be situations when one's CV is completely skipped due to various reasons and thus missing the chance to be employed or have his repositories examined by a professional which could prove otherwise (that one has all the  abilities and skills needed for the position).
\newline
\noindent\textit{Objective:}
This study aims to identify and investigate if its possible to acquire use-full information's from a GitHub profile that could be beneficial for recruiters in a recruitment process. If that would be proven to be an effective way of measuring one's abilities and skills - maybe it could be also used by the developers themselves to attest their own potential quality.
\newline
The visible main challenge is the inefficient hiring process with no accurate technical screening methodology. The goal is to improve the rate of recruiting highly skilled programmers by making use of the information's that are already provided in CV's by standard.
\newline
\noindent\textit{Method:} We conducted a systematic review using a database search in Scopus and evaluated it using the quasi-gold standard procedure to identify relevant studies.
We created polls to gather information from developers own experiences and asked them to attach their GitHub profiles to the questionnaire. Then, with machine-learning based approach, we used that data to automatically evaluate the potential of a candidate by giving only the link to GitHub profile on input.
In the data sheet used to obtain data from publications, we factor the research questions into finer-grained ones, which are then answered on a per-publication basis. Those are then merged over a set of publications using an automated script to obtain answers to the posed research questions.
\newline
\noindent\textit{Results:}
...
\newline
\noindent\textit{Conclusion:}
We conclude that based on the results, we achieved that there is enough data stored on \emph{GitHub} profiles from which machine learning algorithms could provide a reasonably reliable evaluation of the potential quality of the user as an employee, although more research with bigger sample sizes are needed to improve the credibility of these results.
}

\abstract{
\label{sec:abstract}
\newline
\noindent\textit{Context:} Nowadays, it is much harder for new programmers to acquire industry experience. This trend started being noticeable at around 2018, but now - due to the coronavirus pandemic - it is even more prominent. Several dozens of junior programmers apply for one job offer and thus there can be a need for a tool that would facilitate the identification of candidates that the employer is interested in. It is considered a good practice for Junior Developers that are seeking to be hired to have public repositories with some of their personal work. The problem is that usually the recruiter does not have the knowledge to use that information, which - in fact - could be very impactful as it is literally the "show-off" of one's current skills. Right now it is completely disregarded until the later stages of the recruitment phase - after the junior already passes the first recruiter look-up. Only then the repository is examined by a technical interviewer but that also could be a concern for the company as it books out the valuable time of an expertised employee. Moreover, there could be situations when one's CV is completely skipped due to various reasons and thus missing the chance to be employed or have his repositories examined by a professional which could prove otherwise (that one has all the  abilities and skills needed for the position).
\newline
\noindent\textit{Objective:}
This study aims to identify and investigate if its possible to acquire use-full information's from a GitHub profile that could be beneficial for recruiters in a recruitment process. If that would be proven to be an effective way of measuring one's abilities and skills - maybe it could be also used by the developers themselves to attest their own potential quality.
\newline
The visible main challenge is the inefficient hiring process with no accurate technical screening methodology. The goal is to improve the rate of recruiting highly skilled programmers by making use of the information's that are already provided in CV's by standard.
\newline
\noindent\textit{Method:} We conducted a systematic review using a database search in Scopus and evaluated it using the quasi-gold standard procedure to identify relevant studies.
We created polls to gather information from developers own experiences and asked them to attach their GitHub profiles to the questionnaire. Then, with machine-learning based approach, we used that data to automatically evaluate the potential of a candidate by giving only the link to GitHub profile on input.
In the data sheet used to obtain data from publications, we factor the research questions into finer-grained ones, which are then answered on a per-publication basis. Those are then merged over a set of publications using an automated script to obtain answers to the posed research questions.
\newline
\noindent\textit{Results:}
...
\newline
\noindent\textit{Conclusion:}
We conclude that based on the results, we achieved that there is enough data stored on \emph{GitHub} profiles from which machine learning algorithms could provide a reasonably reliable evaluation of the potential quality of the users behind them as employees, although more research with bigger sample sizes are needed to improve the credibility of these results.
}


% |-------------------------------------------------------------------------------------
% | Keywords
% |

% \keywords{github \and machine learning \and predictive modelling \and systematic review \and profiling \and evaluation \and evaluating \and potential \and candidate \and resume \and cv \and job \and market \and recruiters \and screening }

% |-------------------------------------------------------------------------------------
% | Introduction.
% |

\section{Introduction}

While hiring practices of graduates are not thoroughly studied in the context of
the IT industry, the concept in general is well understood
Selecting the most valuable candidate for interview is an important process. While hiring practices of graduates are not thoroughly studied in the context of the IT industry, the concept in general is well understood \cite{HiringProcess}. As a standard, recruiters look through Curriculum Vitae (\emph{CV}) to categorize all candidates and call them for interview. Nowadays, to minimize the time spent on a single CV, recruiters have various pre-screening methods, listing them in order of importance \cite{SantanderCVLectureScholarship}:
\begin{itemize}
  \item employment history and previously business-related work
  \item university and academic performance
  \item listed programming abilities and other colloquially called \emph{"buzzwords"}
  \item soft skills
\end{itemize}

The CV could include a lot of important information's but the recruiter usually spends from 6 to 60 seconds at max per one application \cite{SantanderCVLectureScholarship}. It is surely not enough time to examine candidate skills correctly, more over - the recruiters usually doesn't even have the industrial knowledge to asses the skills and examine candidate's quality by browsing through "code portfolio". It is done only after the pre-screening is already done, when the filtered-out list of candidates goes to a professional technical employee that has the required knowledge and skills to examine the potential of a candidate. The information is there - various online hosting websites that allow to showcase one's code are widely used and added to the CV's - but a qualified candidate may be omitted in the pre-screening method because his cv-building skills were not good enough to pass.

One of such hosting websites is GitHub \footnote{GitHub Website URL: https://github.com/} - it is a service supporting version control using git, which is used by programmers and developers for storing and managing theirs code-base repositories. The popularity of GitHub in 2021 is well established and still rising. Already in 2020 the website gathered over 36 million users \cite{GitHubUsers2020} and today, as of March 2021 - there are 55 million users \cite{GitHubUsers2021}. Dabbish et al. \cite{DabbishC} showed that software developers in public project on GitHub consciously manage their online reputation. They do recognise that others developers can assess their personal characteristics such as the quality of work or commitment by looking for example through their commits and judging the quality, checking the amount of forks of an project or stars. Singer et al. \cite{Singer} found that the amount of followers one have on GitHub is also an valuable information.

We recognise that the CV evaluation procedure has several shortcomings and not much has changed since 2013 \cite{Capiluppi}. These shortcomings especially apply for those who do not have any relevant work experience or formal degree. We also note that GitHub is a place storing lots of very valuable informations that could (or maybe should) be used to evaluate potential of a candidate. Various research works have confirmed that it is possible to retrieve data like technical technical role \cite{TechnicalRole}, identifying Experts in Software Libraries and Frameworks \cite{SoftwareLibraries} .Because the recruiters have very limited time available to review a single CV, we search for an machine-learning based solution that could help extract the information from GitHub by a non-expertised recruiter. This could not only reduce the amount of false-positive skips but also increase the overall quality of list of candidates after the pre-screening is done.


% |-------------------------------------------------------------------------------------
% | Systematic Review.
% |
% | - Research Questions
% | - Currently Available Resources on the Subject
% | - Results Selection Process

\section{Systematic Review}
\label{sec:systematic-review}

What distinguishes this paper among others — as we mentioned — we do recognize that the majority of recruiters do not have skills that could examine the technicality of a candidate. For example in the \emph{"Mining the Technical Roles of GitHub Users"} the survey was conducted among recruiters from StackOverflow website \footnote{StackOverflow Website URL: \url{https://stackoverflow.com/}} - which is primarily used by developers for the purposes of collaboration \& knowledge sharing. Thus, it can be concluded that the surveyed sample was exceptional among the global population of recruiters as they have made extra efforts to create an account on IT-technical website and not only that - they were actively browsing it and even possibly creating extra content which made them visible and reachable by researchers. Based on the research we have conducted, there are only a few unfinished recommendation systems that use the user GitHub profile, but at the same time there is some research that focuses on mining data from the repositories.

\subsection{Research Questions}
Research questions presented below should explain to help understand the importance of creating such a tool and provide information about the functionalities which the potential end user might need (i.e. - recruiters, developers themselves). Furthermore, we also explore issues related to the recruitment itself, as well as the quality of repositories.

Hence, we defined four research questions.
\begin{itemize}
\item{\emph{RQ1: What are the most common skills the company is looking for among the candidates?}\\
  Our system needs to determine whether the developer is worthy of attention and if he should be considered further in the recruitment process. The answer to this question should explain what skills are considered important among companies. }
\item{\emph{RQ2: How a given repository makes a good impression?}\\
  One way of assessing developer expertise is to analyse his projects. Thus, we should determine which features of repositories are the most representative ones.} 
\item{\emph{RQ3: Is it possible to get a good enough information about the candidate's potential based on the GitHub profile?}\\
  This question should provide us with information whether the recommendation system based on repositories is worth devoting time.} 
\item{\emph{RQ4: Is GitHub repository used only for programming purposes?}\\
  It often happens that an item is used contrary to the original intention of the author. The same situation might occur with the repositories. There is no compulsion to prohibit the use of a repository for purposes other than development.}
\end{itemize}

\subsection{Currently Available Resources on the Subject}
We used search engines (Google Scholar and Scopus) to find proper resources and currently available papers on the subject that we are investigating in our paper, and then - to collect all relevant literature - we created a more specific search string which is given below\textit{ [ref: ~\ref{lst:search-string}]}.

% This is final v1.0.0 formatting - it looks better if the page breaks here // MJ %
\newpage

\begin{lstlisting}[language=Query, label={lst:search-string}, caption=Search String Query]
(
  skill AND
  repository AND
  ( github OR gitlab OR bitbucket )
) OR (
  recruitment OR hiring OR interview
) AND (
  programming AND purpose
) AND (
      "Computer Science"
) AND (
  mining AND software
)
\end{lstlisting}
Because of the enormous number of results in Scopus, we decided to add another limitation such as the required presence of ,,Software Engineering'' keyword and ,,2017-2021'' year constraint to narrow it down in time to only the most relevant studies.

This gave us a good amount of publications which needed to be further filtered by hand.  As shown on the literature identification figure~\textit{[ref: \ref{fig:literature-identification}]} search engines provided us with over two hundred publications. Unfortunately, many of them did not answer the previously established research questions and many of them were simply not in English nor Polish language so that we can understand them. After all, we have identified ten of the most important studies that are helpful in our topic.

\begin{figure}[htp]
\centering
\includegraphics[width=\linewidth]{img/search_process_new.png}
\caption{Literature Identification}
\label{fig:literature-identification}
\end{figure}

% This is final v1.0.0 formatting - it looks better if the page breaks here // MJ %
\newpage

\subsection{Results Selection Process}
After running the search string query, we were presented with a large number of results. We have gone through the results list and discarded all duplicate entries and simultaneously discarded those papers whose titles or abstracts were not related to the topic of our research. Thus, we ended up with 28 papers which we further examined. Finally, we have chosen 10 papers which are closely related to the subject of this paper.


% |-------------------------------------------------------------------------------------
% | Literature Review.
% |

\section{Literature Review}
Choosed literature gives us answears to the questions asked earlier in this paper and provide us some useful informations. \\Skills which are needed from IT specialists descirbes \cite{StackOverflowStudies} based on job offers posted on Stack Overflow. They extract needed data and then define which hard and soft skills are welcome among recruiters. This study helps us to depict usefull skills and also can help young programmers to find their first dream job as a software engineer. Paper \cite{DoOnBoardingProgramsWork} gives us a very interesting look on boarding programs. It aware us that not every contribiutor to the open source systems is going to be a skilled programer. Based on this article many contributors are people who took part in a boarding program which are mentored by some master. Master gives participants many tips and helps them with their problems. Having that information we should be careful looking at repository contributors because not everyone is going to be professional. Article \cite{GitHubProfilesToJobAdv} shows us how skills extracted from job offers can be matched with developers skills extracted from repositories. Description of data scraping and how matching can be done should be useful in system which is presented in this paper.
\\Subject of repositories is discussed in \cite{MiningGitHub}, they take a deep look into the structure and usage of repositiories. This paper points out that not every repository is used for programming porpuses. Many of the repositories are abandonded few weeks after creation which causes very large actvity in first weeks but with the time passing, activity is decreasing drastically. A good amount of peple are using repositories as a storage or they simply have empty repositories as an effect of some experiments or attempts of learning. 
\\Recruitment process is described by the developers in paper \cite{HiringIsBroken} where authors collected data from actuall developers who posted their opinions on social networks. There is said that recruiment process casues a lot of stress and is not liked among developers. Programmers say that during the interview they can not show their whole potential and they are served with excercises which are not useful during normal work and do not check their actuall knowleadge. Developers criticise big companies for their ease of rejecting canditates beacause there are going to be much more candidates and it is not going to be a big mistake if they reject someone good - simply they will find someone else.
\\In recent years there have been carried out a various similar types of research concerning issues related to our study. For instance, there was created an online developer profiling tool assessing developer expertise \cite{GitLabProfilingTool}. The tool considers factors such as code quality, code quantity, contribution to rank  software engineer’s repositories and his skills as well. Evaluations of developers using greater number of features were determined as more accurate ones. However, weight of all of the indicators was equivalent, so none of them could be chosen as the most significant. Moreover, the tool is based on analysis of repositories created on GitLab which is definitely less popular than GitHub \cite{GitLabvsGithub}.
\\Furthermore, a relation between technical and social skills \cite{WhatMakesGoodDev} was explored also. In the study which aims at finding out what makes developer good we discovered that there is a lack of strong association between above types of skills, so doubtless it is extremely tough to assess soft and social skills mining GitHub repositories.
\\Generally, candidate selection process might be based on LinkedIn profile, CV and Github Profile as well \cite{CandidateSelection}. Although it could be complicated to obtain information about non-technical skills using above resources, personality traits of candidates could be accessed from analyzed transcription of a phone call.
\\In terms of identifying experts in software libraries and frameworks \cite{SoftwareLibraries} it was analyzed which features best distinguish library specialists. First of all, it was explored that the feature values are different for experts in each library. Nevertheless, the features such as number of commits are always relevant.


% |-------------------------------------------------------------------------------------
% | Methodology.
% |
% | - Data Collection
% | - Data Preprocessing
% | - Creating Machine Learning Model

\section{Methodology}
In this section we defined methods, techniques, tools and processes referring to the main part of our research.

\subsection{Data Collection}
\label{sec:data-collection}

\subsubsection{Questionnaire}
\label{sec:questionnaire}

Our main goal during the data collection process was to obtain GitHub Usernames along with as many answers to questions related to the subject of the recruitment process, an assessment of own skills as well as some additional information which could help in the reconstruction of results found in other papers. We have created a Google Form questionnaire and then posted it on two programming-related Facebook Groups.

In both groups, one can find people with a full range of professional experience but based on the activity (posts and comments), one of them is mainly characterized by highly qualified employees (Seniors) while the other seems evenly distributed.

\subsubsection{Parsing General GitHub Information}
\label{sec:github-info-parsing}

To collect data from GitHub, we used GraphQL Queries \footnote{GraphQL GitHub API: \url{https://docs.github.com/en/graphql}}. The query takes only the username on the input and on the output puts only the information precisely specified within the query itself. GraphQL Query is presented in the image below ~\textit{[ref: \ref{fig:graph-ql-query}]}.

\begin{figure}[htp]
\centering
\includegraphics[width=11cm]{query}
\caption{GraphQL Query}
\label{fig:graph-ql-query}
\end{figure}


Requests were done within R script for every username, which was extracted from a GitHub link that was provided in the questionnaire. Unfortunately, not every participant provided a relevant link to the repository or even did not provide any. First, the script loads the questionnaire data and then for every username executes the request to \emph{GitHub API}. In case of an invalid link, it returns nothing. After collecting data for every user who took part in the questionnaire, the script combines the survey data with the information retrieved via \emph{GitHub API} and saves it as \code{csv} file.


\subsubsection{Parsing GitHub Repositories}
\label{sec:github-repo-parsing}

This part requires quite a bit of time to complete per just one single user. The time it takes to fetch and parse is mainly related to the number of repositories that a given user has as well as the contents which are in them. The script receives on input one single username and then it collects all repository names into a list. In the next step, all of these repositories are fetched from remote to the local computer so they can be checked via \emph{Mega Linter}. After collecting all of the repositories, \emph{Mega Linter} task is executed in each single one of them. When that is done, the script moves on to the merge task, which transforms all the generated \code{jsons} into one, single \code{json} file, so that it can be easily imported and accessed by the model in \emph{R}.




\subsection{Data Preprocessing}
\label{sec:data-preprocessing}

\subsubsection{Questionnaire}
\label{sec:data-prep-questionnaire}

In total, we received 67 completed questionnaires and 45 of them were provided with GitHub profile. The data file was further cleaned and reformatted with the use of Python \footnote{Python Website: \url{https://www.python.org/}} script. Both the original and formatted \code{csv} files can be found in the project source under the name ,,\code{questionnaire.csv}'' and ,,\code{cleaned\_data.csv}'' as well as the Python script used for data formatting which can be found in ,,\code{py\_scripts}'' directory under the name ,,\code{questionnaire\_adjuster.py}''. To run the script, one must have Python 3.9 or newer installed on the machine along with ,,Numpy'' \footnote{Numpy PyPI Page - \url{https://pypi.org/project/numpy/}}, ,,Pandas'' \footnote{Pandas PyPI Page - \url{https://pypi.org/project/pandas/}} and ,,Requests'' \footnote{Requests PyPI Page - \url{https://pypi.org/project/requests/}} packages.

Delving into the details, starting with the most important - 7 answers were manually modified to standardize the values into one datatype (\code{int}), as they were given in the questionnaire as strings, so that people are not constraint to the few selected numerical values and - unfortunately - there was no other way to have an input field that allows only for \code{UnlimitedNatural} numbers. Majority of these answers were given as ranges (ex.: $10 - 20$, \emph{a dozen}, $30-40$) - these were modified to be the average of the given range. The "self-doubtable" and the ,,not exactly sure'' answers like $60?$, $4/5$ and $20+$, were changed to the lowest value in the suggested range. 

Besides that - all available checkbox answers (\emph{Soft Skills} \& \emph{Pre-work Experience}) were transformed so that each of them has their own respective column with \code{Boolean} answers whether it was checked or not.

Other changes are technical and linguistic so that the data set could be reused with comprehension without the knowledge of Polish language:

\begin{itemize}
  \item Unicode Characters (like \emph{Emojis}) were removed from the column names, and the column names themselves are now English words
  \item Every single column had their data properly adjusted so they could be interpreted with their suitable data type
  \item Similar columns from different sections of the questionnaire which were the result of answers to similar but rephrased questions merged
  \item Answers which were given as strings like ,,\code{3,5 year}'' were transformed into \code{integers} in units of months
\end{itemize}

\subsubsection{Repositories}
\label{sec:data-prep-repositories}

One of the most difficult tasks is to properly obtain information from the repositories of a given user. For research purposes, we decided to track them for potential errors and warnings which can be identified via linters. Linter is a static code analysis tool used to flag any bugs, errors, stylistic warnings, suspicious constructs, redundant code, and more depending on the language and/or tool. Of course, the user could have repositories with code written in any language that exists, and that is a real problem, which we mitigated by a linter-aggregating tool - \emph{Mega Linter} \footnote{Mega Linter GitHub Page - \url{https://github.com/nvuillam/mega-linter}}. Mega Linter is an open-source tool that simply detects the languages used in a given project and then uses all available linters to scan it through. After the scan, it prints out a summary table with the number of Files that were detected and scanned with a given linter, the number of fixed files automatically during the run time, and the number of errors that couldn't be automatically fixed. There's also a second table that's printed somewhere in the first half of the output log that has information about detected duplicate lines and tokens in the project. To obtain that data we redirected the output stream into a file and then parsed it with Python script ,,\code{scrape.py}'' which can be found in \emph{,,py\_scripts''} folder.

The usage of the Mega Linter and scrape scripts is more widely described in the main \code{README.md} file although, in short, one must have Docker \footnote{Docker Website - https://www.docker.com/} and Python installed. Then, simply navigate to the repository which you would like to lint and run a command \textit{[ref: \ref{lst:shell-command-run-linter}]} which will generate an \code{output.txt} file.

\begin{lstlisting}[language=BashOwn, label={lst:shell-command-run-linter}, caption={"Running
\emph{Mega Linter}"}]
npx mega-linter-runner --flavor all
                       -e 'ENABLE=,DOCKERFILE,MARKDOWN,YAML'
                       -e 'SHOW_ELAPSED_TIME=true'
                       > output.txt
\end{lstlisting}

Copy the file and paste it in the \emph{scrape.py} directory. Finally use shell scrape command \textit{[ref: \ref{lst:shell-command-scrape-py}]} command which will generate an \code{output.json} with a list of dictionaries structured as in output example given below \textit{[ref: \ref{lst:scrape-py-output-example}]}.

\begin{lstlisting}[language=BashOwn, label={lst:shell-command-scrape-py}, caption={"Launching \code{scraper.py}"}]
python scraper.py -f output.txt
\end{lstlisting}

\begin{lstlisting}[language=PythonOwn, label={lst:scrape-py-output-example}, caption={"Parsed Linter Output in \code{.json} format"}]
{
 "language": str,     # detected language
 "linter": str,       # checked via linter (name)
 "files": int or str, # detected files in given language
 "fixed": int,        # fixed errors automatically
 "errors": int        # errors that could not be fixed
},

# or

{
 "language": str,     # detected language
 "files": int,        # in a given language
 "lines": int,        # in a given language
 "tokens": int,       # ("chars") in a given language
 "clones": int,
 "duplicate_lines_num": int,
 "duplicate_lines_percent": float,
 "duplicate_tokens_num": int,
 "duplicate_tokens_percent": float
},
\end{lstlisting}


\subsection{Creating a Machine Learning Model}
\label{sec:creating-machine-learning-model}

We decided to chose \code{mlr} \footnote{mlr Website: https://mlr.mlr-org.com/} package as a tool which assists in creation of machine learning models and which is by far one of the most convenient machine learning packages in R. \footnote{R Website: https://www.r-project.org/}. We have performed the following processes:

As our first step, we began by importing the data from the \code{csv} file \footnote{The \code{csv} file is available at the ,,\code{./data/cleaned_data.csv}'' path in this project repository} with appropriate  \code{UTF-8} encoding into a data frame.
Then, in our first approach to the task, we have split the data into two separate data sets, one of which is the training set which consist of 75\% of our data and the other - testing set with the remaining rest. 

Now, after the data set is ready, we can proceed towards the creation of the machine learning model. First, we have defined a learning task for classification. We have specified \code{Task ID}, \code{Process Data} and \code{Target Column} (target column holds the information whether a given user is interesting in terms of his potential as an employee - or in other words - ,,attention worthy''). 

%%% OLD \start %%%
Subsequently, we constructed a learner by calling \emph{makeLearner()} method. We needed to choose a classification algorithm. In the beginning, \emph{randomForest} was selected. Afterwards, we trained the model with a \emph{train()} method. We were supposed to specify train subset, use defined task and learner. In the end, we predicted the target values for test dataset. We had to specify using machine learning model and task with a \emph{predict()} method as well. Moreover, by calling \emph{makeResampleDesc()} method we determined a resampling strategy used in our process. We selected a cross-validation as our strategy in order to estimate precisely how accurately a predictive model will perform in practice. Initially, we set 10-fold cross-validation type which means we used cross-validation with 10 iterations.
%%% OLD \end %%%


This was the time to begin the most rewarding part of the study. Having collected the data from all previously mentioned sources (which is the survey among programmers and IT specialists \textit{[ref: \ref{sec:questionnaire}, \ref{sec:data-prep-questionnaire}]}, the mega linter scans of repositories of the surveyed developers \textit{[ref: \ref{sec:github-repo-parsing}]} and the various information available through the GitHub API \textit{[ref: \ref{sec:github-info-parsing})]} into one single \code{csv} file, we began the work. 

First, we filtered out those users which did not provide their GitHub username and those who did not classify any ,,main'' language (and two more users whose repositories could not get linted because of the limited available time to perform such operations or one more user who had no public repositories on his GitHub profile). The remaining pool of <ILE ZOSTAŁO KONIEC KOŃCÓW USERÓW?> was the base from which we extrapolated the features that were used in the machine learning model:

\begin{itemize}
  \item \code{ExperienceDuration} - The time that a person is being present on the labor market.
  \item \code{WorkFindTime} - The time that elapsed until the first job by a given person was found.
  \item \code{AvgCommitTime} - The average time that between commits. Missing values were filled by inserting the averages.
  \item \code{ExpType} - Weighted average of the type of experience the person has.
  \item \code{SoftSkills} - Weighted average of the soft skills that a person had listed on their CV, in which the most important factor of $2$ was given to the \emph{communicativeness}, while the rest is set as $1$. 
  \item \code{DupLinesPercent} - The percent of lines that were detected as duplicated throughout a given repository summed over all repositories.
  \item \code{LinesPerFile} - The ratio that is calculated via dividing of the number of lines by the number of files.
\end{itemize}

The data was labeled on the basis of current analysis of available information about the users. Then, we proceed to analyse the correlations between variables in that data.


% |-------------------------------------------------------------------------------------
% | Results.
% |

\subsection{Results}
\label{sec:results}

We decided to compare the results for three different models. Below you can find two tables - one of them is based on the results which were achieved using all the available data sources we have (which includes the questionnaire) \textit{[ref: table\ref{table:1}]}  while the other uses only the data that can be gathered using GraphQL request to GitHub API and the Mega Linter with parsing scripts \textit{[ref: table\ref{table:2}]}. In the second case, we used \code{AvgCommitTime}, \code{DupLinesPercent} and \code{LinesPerFile} attributes. Below the tables, we present the correlation matrix of individual features \textit{[ref: figure\ref{fig:correlation-matrix}]}. 

\begin{table}[h!]
\centering
\begin{tabular}{ | m{5em} | m{5em}| m{5em} | m{5em}| m{5em} | m{5em}| } 
\hline
 Algorithm & \textbf{MMCE} & \textbf{MCC} & \textbf{F1} & \textbf{ACC} & \textbf{Kappa}\\ 
 \hline
 \textbf{Random Forest} & 0.329 & 0.245 & 0.410 & 0.671 & 0.219 \\ 
\hline
 \textbf{SVM} & 0.315 & 0.356 & 0.422 & 0.685 & 0.311 \\ 
\hline
 \textbf{KNN} & 0.388 & 0.224 & 0.431 & 0.611 & 0.184\\ 
\hline
\end{tabular}
\caption{Measures scores for each classifiers (all features)}
\label{table:1}
\end{table}

\begin{table}[h!]
\centering
\begin{tabular}{ | m{5em} | m{5em}| m{5em} | m{5em}| m{5em} | m{5em}| } 
\hline
 Algorithm & \textbf{MMCE} & \textbf{MCC} & \textbf{F1} & \textbf{ACC} & \textbf{Kappa}\\ 
 \hline
 \textbf{Random Forest} & 0.360 & 0.299 & 0.451 & 0.639 & 0.245 \\ 
\hline
 \textbf{SVM} & 0.319 & 0.314 & 0.451 & 0.680 & 0.280 \\ 
\hline
 \textbf{KNN} & 0.367 & 0.237 & 0.484 & 0.632 & 0.209\\ 
\hline
\end{tabular}
\caption{Measures scores for each classifiers (\emph{GitHub API} and \emph{Mega Linter} features)}
\label{table:2}
\end{table}

\begin{figure}[htp]
\centering
\includegraphics[width=\linewidth]{r_plot_corrrelation.png}
\caption{Correlation Matrix}
\label{fig:correlation-matrix}
\end{figure}


% |-------------------------------------------------------------------------------------
% | Discussion.
% |

\section{Discussion}
\label{sec:discussion}

\subsection{All Features}

The variables prepared for the machine learning model were not correlated with each other. No substantial correlation was observed between the percentage of duplicate lines and the time until the first job was found. Unnoteworthy although reassuring connection, which has the highest correlation among those all variables, between the experience duration and \emph{WorkFindTime} is present. 

Taking into account the results of the most common classifiers (\emph{ACC} and \emph{MMCE}) we conclude that the \emph{Random Forest} and \emph{SVM} algorithms coped best with the classification task. The \emph{SVM} had the best results, which is not a surprise - as in theory - it should produce solid results even with a small amount of training data. That also reassures the validity of this algorithm as it is ,,field-tested'' in other research papers with similar data size with success. \emph{KNN} which by definition is the simplest, performs the worst because it requires features in a uniform scale to classify with its best efficiency while our data has features of quite different scales.

However, one should have doubts about the quality of these indicators (the \emph{ACC} and \emph{MCEE}) because the data that we feed into the algorithm is not perfectly balanced (the $\%$ share of classes in the set is about $62\%$ to $38\%$). The \emph{F1} measure assessed the quality of all models on comparable level. On the other hand, the \emph{MCC}, which is often considered a key indicator, showed that the \emph{SVM} model was by far the best in classifying tasks.

The \emph{Kappa} coefficient for the \emph{Random Forest} and \emph{SVM} algorithms is in the range of $0.21 - 0.40$, therefore the \emph{Level of Agreement} can be describe as \emph{,,fair agreement''} (which is about an average result). For \emph{KNN} the compliance can be defined as \emph{,,slight agreement''}.

To sum up, the most effective model during the classification was the one using the \emph{SVM} algorithm. It was the best according to almost all indicators.

\subsection{GitHub Data Only}

Comparing the results of the classification using different features, we observed that a different number of features influenced each model in a different way. Each of the measures in the model that used \emph{SVM} algorithm, has worse value (besides the \emph{F1} indicator, which is better for each algorithm). This indicates that the algorithm performs better when it has access to the higher variety of features. 

The \emph{KNN} algorithm performance improved as the number of features decreased. This may be due to the fact - as we previously mentioned - that it should perform best with the features of similar scale. As for \emph{Random Forest} - lower number of features made the \emph{ACC} indicator significantly worse while improving \emph{MCC}. It is difficult to explain the reason why such a difference in the results was obtained in this particular case.


\subsection{Run-down}

Summarizing, in most cases, the results of classification based on fewer features are worse, but the difference is not dramatic. Therefore, it can be concluded that the data obtained from the repository using our method of combining \emph{GitHub API} along with \emph{Mega Linter} tool and our parse scripts is to some extent sufficient to assess the potential of the GitHub user in terms of his potential for an employee.


% |-------------------------------------------------------------------------------------
% | Conclusions.
% |

\subsection{Conclusions}
\label{sec:conclusions}

% |-------------------------------------------------------------------------------------
% | Acknowledgement & Bibliography.
% |

\begin{acknowledgement}
\item {This research was ...}
\end{acknowledgement}

\bibliographystyle{plain}
\bibliography{refs}

\end{document}
