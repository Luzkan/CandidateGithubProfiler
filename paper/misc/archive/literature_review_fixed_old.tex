\section{Literature Review}
Chosen literature gives us answers to the questions asked earlier in this paper and provides us some useful pieces of information. \\Skills which are needed from IT specialists describes \cite{StackOverflowStudies} based on job offers posted on Stack Overflow. They extract needed data and then define which hard and soft skills are welcome among recruiters. This study helps us to depict useful skills and also can help young programmers to find their first dream job as software engineers. Paper \cite{DoOnBoardingProgramsWork} gives us a very interesting look onboarding programs. We are aware that not every contributor to the open-source systems is going to be a skilled programmer. Based on this article many contributors are people who took part in a boarding program which are mentored by some master. Master gives participants many tips and helps them with their problems. Having that information we should be careful looking at repository contributors because not everyone is going to be professional. Article \cite{GitHubProfilesToJobAdv} shows us how skills extracted from job offers can be matched with developers' skills extracted from repositories. Description of data scraping and how matching can be done should be useful in a system that is presented in this paper.
\\Subject of repositories is discussed in \cite{MiningGitHub}, they take a deep look into the structure and usage of repositories. This paper points out that not every repository is used for programming purposes. Many of the repositories are abandoned few weeks after creation which causes very large activity in the first weeks but with time passing, activity is decreasing drastically. A good amount of people are using repositories as storage or they simply have empty repositories as an effect of some experiments or attempts of learning. 
\\Recruitment process is described by the developers in paper \cite{HiringIsBroken} where authors collected data from actual developers who posted their opinions on social networks. There is said that the recruitment process causes a lot of stress and is not liked among developers. Programmers say that during the interview they can not show their whole potential and they are served with exercises that are not useful during normal work and do not check their actual knowledge. Developers criticize big companies for their ease of rejecting candidates because there are going to be much more candidates and it is not going to be a big mistake if they reject someone good - simply they will find someone else.
\\In recent years there have been carried out various similar types of research concerning issues related to our study. For instance, there was created an online developer profiling tool assessing developer expertise \cite{GitLabProfilingTool}. The tool considers factors such as code quality, code quantity, contribution to rank software engineer’s repositories, and his skills as well. Evaluations of developers using a greater number of features were determined as more accurate ones. However, the weight of all of the indicators was equivalent, so none of them could be chosen as the most significant. Moreover, the tool is based on analysis of repositories created on GitLab which is less popular than GitHub \cite{GitLabvsGithub}.
\\Furthermore, a relation between technical and social skills \cite{WhatMakesGoodDev} was explored also. In the study which aims at finding out what makes developers good, we discovered that there is a lack of strong association between the above types of skills, so doubtless it is extremely tough to assess soft and social skills mining GitHub repositories.
\\Generally, the candidate selection process might be based on LinkedIn profile, CV, and Github Profile as well \cite{CandidateSelection}. Although it could be complicated to obtain information about non-technical skills using the above resources, the personality traits of candidates could be accessed from the analysed transcription of a phone call.
\\In terms of identifying experts in software libraries and frameworks \cite{SoftwareLibraries} it was analysed which features best distinguish library specialists. First of all, it was explored that the feature values are different for experts in each library. Nevertheless, features such as the number of commits are always relevant.
