
\subsection{Creating a Machine Learning Model}
Essentially, we decided to chose mlr \footnote{mlr Website: https://mlr.mlr-org.com/} package as a tool which is supposed to help us in creating a machine learning model and which is one of the best machine learning packages in R. \footnote{R Website: https://www.r-project.org/}We determined that we wanted to perform a binary classification, hence we performed the following processes

Firstly, we imported data from csv file with appropriate encoding. We saved selected columns from imported data into a data frame as well. Then, as a first approach we split the data into two datasets in a given ratio: training set - 75\% of data, test set – 25\% of data. Finally, the datasets were prepared and ready to use. The first step toward creating a machine learning model was to define a learning task for a classification. We specified task ID, data used in process and a target column, which had to be a factor, with a \emph{makeClassifTask()} method. Subsequently, we constructed a learner by calling \emph{makeLearner()} method. We needed to choose a classification algorithm. In the beginning, \emph{randomForest} was selected. Afterwards, we trained the model with a \emph{train()} method. We were supposed to specify train subset, use defined task and learner. In the end, we predicted the target values for test dataset. We had to specify using machine learning model and task with a \emph{predict()} method as well. Moreover, by calling \emph{makeResampleDesc()} method we used a cross-validation in order to estimate precisely how accurately a predictive model will perform in practice.
