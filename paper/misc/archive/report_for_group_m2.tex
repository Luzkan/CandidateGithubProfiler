%%%%%%%%%%%%%%%%%%%% author.tex %%%%%%%%%%%%%%%%%%%%%%%%%%%%%%%%%%%
%
% sample root file for your "contribution" to a contributed volume
%
% Use this file as a template for your own input.
%
%%%%%%%%%%%%%%%% Springer %%%%%%%%%%%%%%%%%%%%%%%%%%%%%%%%%%


% RECOMMENDED %%%%%%%%%%%%%%%%%%%%%%%%%%%%%%%%%%%%%%%%%%%%%%%%%%%
\documentclass[graybox]{svmult}

% choose options for [] as required from the list
% in the Reference Guide

\usepackage{bibentry}
\usepackage{type1cm}        % activate if the above 3 fonts are
                            % not available on your system
%
\usepackage{makeidx}         % allows index generation
\usepackage{graphicx}        % standard LaTeX graphics tool
                             % when including figure files
\usepackage{multicol}        % used for the two-column index
\usepackage[bottom]{footmisc}% places footnotes at page bottom

\usepackage{newtxtext}       %
\usepackage{newtxmath}       % selects Times Roman as basic font
\usepackage{dirtytalk} \newcommand{\mysay}[1]{\say{\textit{#1}}}
\usepackage{enumerate}
\usepackage[unicode,colorlinks=true,breaklinks,allcolors=black]{hyperref}
\usepackage{cleveref}
\usepackage{ltablex}
\usepackage{booktabs}
\usepackage{makecell}
\usepackage{doi}
\usepackage{enumitem}

% Probably should be swapped for JPEGs!
\usepackage{pgfplots}
\usepgfplotslibrary{statistics}
\usetikzlibrary{pgfplots.statistics} % LATEX and plain TEX
\usetikzlibrary[pgfplots.statistics] % ConTEXt

%\nobibliography
\usepackage{fixme}

\nobibliography*


% see the list of further useful packages
% in the Reference Guide

\makeindex             % used for the subject index
                       % please use the style svind.ist with
                       % your makeindex program

%%%%%%%%%%%%%%%%%%%%%%%%%%%%%%%%%%%%%%%%%%%%%%%%%%%%%%%%%%%%%%%%%%%%%%%%%%%%%%%%%%%%%%%%%



\begin{document}

\title*{Sprawozdanie z Milestone'u \#1 Grupy M2}
\author{Marcel Jerzyk, Jakub Litkowski, Jakub Szańca}
\institute{
Marcel Jerzyk \at Wroclaw University of Science and Technology, Poland, \email{244979@student.pwr.edu.pl}
\and Jakub Litkowski \at Wroclaw University of Science and Technology, Poland, \email{242353@student.pwr.edu.pl}
\and Jakub Szańca \at Wroclaw University of Science and Technology, Poland, \email{242519@student.pwr.edu.pl}
}

\maketitle



\newpage

\section{Performance Test}

Uruchomienie programu przy pomocy komendy z \textit{README.md} ,,\textit{mvn exec:java}'' trwało:



\begin{table}[!h]
\centering
        
\begin{tabular}{|p{0.33\textwidth}|p{0.33\textwidth}|p{0.33\textwidth}|}
\hline 
 \begin{center}
\textbf{Komputer}
\end{center}
 & \begin{center}
\textbf{Próba}
\end{center}
 & \begin{center}
\textbf{Czas}
\end{center}
 \\
\hline 
 \multirow{$\displaystyle k_{1}$} & $\displaystyle 1$ & 13:52 \\
\cline{2-3} 
   & $\displaystyle 2$ & 13:42 \\
\cline{2-3} 
   & $\displaystyle 3$ & 13:37 \\
\hline 
 \multirow{$\displaystyle k_{2}$} & $\displaystyle 1$ & 24:47 \\
\cline{2-3} 
   & $\displaystyle 2$ & 24:51 \\
\cline{2-3} 
   & $\displaystyle 3$ & 25:02 \\
\hline 
 \multirow{$\displaystyle k_{2}$} & $\displaystyle 1$ & 56:28 \\
\cline{2-3} 
   & $\displaystyle 2$ & --- \\
\cline{2-3} 
   & $\displaystyle 3$ & --- \\
 \hline
\end{tabular}
        
\end{table}

Przy czym konfiguracja komputerów $k_{1}, k_{2}, k_{3}$ jest następująca:

\begin{table}[!h]
\centering
        
\begin{tabular}{|p{0.20\textwidth}|p{0.20\textwidth}|p{0.20\textwidth}|p{0.20\textwidth}|p{0.20\textwidth}|}
\hline 
 \begin{center}
\textbf{Komputer}
\end{center}
 & \begin{center}
\textbf{CPU}
\end{center}
 & \begin{center}
\textbf{GPU}
\end{center}
 & \begin{center}
\textbf{OS}
\end{center}
 & \begin{center}
\textbf{RAM}
\end{center}
 \\
\hline 
 $\displaystyle k_{1}$ & Ryzen 3600 & RTX 2060 2GB & Windows 10 & 32GB @ 1600MHz \\
\hline 
 $\displaystyle k_{2}$ & Intel Core i7  6820HQ & Intel HD Graphics 530  & Windows 10 & 16GB @ 1600 MHz \\
\hline 
 $\displaystyle k_{3}$ &Intel Core i5-7200U  &Intel HD Graphics 620  & Windows 10 & 12GB @ 2133MHz  \\
 \hline
\end{tabular}
        
\end{table}

Otrzymane wyniki w przypadku z każdego testu dla komputera $k_{1}$ były reprodukowalne (identyczne przy każdej próbie).

\begin{frame}{}
\hbox{\hspace{-5.0em} \includegraphics[scale=0.27]{img/results.png}}
\end{frame}



\section{Uwagi}

\subsection{\textit{README.md}}

Plik \textit{README.md} ma bardzo lakoniczny opis, który ujmuje tylko samą metodę aplikacji. Brakuje w nim informacji na temat tego, co się dokładnie dzieje po uruchomieniu danej komendy i w jakim celu się ją wykonuje. Brakuje informacji na temat rezultatu otrzymanych danych oraz ich definicji, aby móc zrozumieć istotę programu. Brak informacji o badanych meta klasyfikatorach. Brak zrzutu ekranu z przykładowych wyników.

Technicznie - brakuje w nim informacji na jakiej wersji \emph{Javy} można uruchomić program, bądź na jakiej był tworzony, a więc na jakiej powinien być testowany. Nie ma informacji na temat użytych bibliotek i ich wersji.

\subsection{Kod}

Kod aplikacji jest zrozumiały oraz dobrze podzielony na pakiety.

\subsection{Report}

Prezentujemy zaimportowane dane z wygenerowanego raportu w programie Excel. \\

\begin{frame}{}
\hbox{\hspace{-5.0em} \includegraphics[scale=0.45]{img/results_loaded.png}}
\end{frame}
%\input{references}
\end{document}
